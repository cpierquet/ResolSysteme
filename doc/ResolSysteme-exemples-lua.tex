% !TeX TXS-program:compile = txs:///lualatex/[--shell-escape]

\documentclass[french,a4paper,10pt]{article}
\usepackage[margin=1.5cm]{geometry}
\usepackage[executable=python.exe]{pyluatex}
\usepackage[lua]{ResolSysteme}
\usepackage{systeme}
\usepackage{codehigh}
\sisetup{locale=FR,output-decimal-marker={,}}
\usepackage{babel}

\begin{document}

\part*{Version \og LUA \fg{} avec pyluatex}

\section{Préambule, sans le package pyluatex}

\begin{codehigh}
\documentclass[french,a4paper,10pt]{article}
\usepackage[margin=1.5cm]{geometry}
\usepackage[executable=python.exe]{pyluatex}
\usepackage[lua]{ResolSysteme}              %version pyluatex, lua + shell-escape
\usepackage{systeme}
\sisetup{locale=FR,output-decimal-marker={,}}
\end{codehigh}

\section{Inverse d'une matrice, 2x2 ou 3x3 ou 4x4}

\begin{demohigh}
L'inverse de $A=\begin{pNiceMatrix} 1&2 \\ 3&4 \end{pNiceMatrix}$ est
$A^{-1}=\MatriceInverseLUA<cell-space-limits=2pt>(1,2;3,4)$.
\end{demohigh}

\begin{demohigh}
L'inverse de $A=\begin{pNiceMatrix} 1&2 \\ 3&6 \end{pNiceMatrix}$ est
$A^{-1}=\MatriceInverseLUA<cell-space-limits=2pt>(1,2;3,6)$.
\end{demohigh}

\begin{demohigh}
L'inverse de $A=\begin{pNiceMatrix} 1&2 \\ 3&4 \end{pNiceMatrix}$ est
$A^{-1}=\MatriceInverseLUA*[d]<cell-space-limits=2pt>(1,2;3,4)$.
\end{demohigh}

\begin{demohigh}
L'inverse de $A=\begin{pNiceMatrix} 1&2&3\\4&5&6\\7&8&8 \end{pNiceMatrix}$ est
$A^{-1}=\MatriceInverseLUA<cell-space-limits=2pt>(1,2,3;4,5,6;7,8,8)$.
\end{demohigh}

\begin{demohigh}
L'inverse de $A=\begin{pNiceMatrix} 1&2&3\\4&5&6\\7&8&8 \end{pNiceMatrix}$ est
$A^{-1}=\MatriceInverseLUA*<cell-space-limits=2pt>(1,2,3;4,5,6;7,8,8)$.
\end{demohigh}

\begin{demohigh}
L'inverse de $A=\begin{pNiceMatrix} 1&2&3&4\\5&6&7&0\\1&1&1&1\\2&-3&-5&-6 \end{pNiceMatrix}$
est $A^{-1}=\MatriceInverseLUA*[n]<cell-space-limits=2pt>(1,2,3,4;5,6,7,0;1,1,1,1;2,-3,-5,-6)$.
\end{demohigh}

\pagebreak

\section{Résolution d'un système, 2x2 ou 3x3 ou 4x4}

\begin{demohigh}
La solution de $\systeme{-9x-8y=-8,3x-6y=-7}$ est $\mathcal{S}=%
\left\lbrace \SolutionSystemeLUA(-9,-8;3,-6)(-8,-7) \right\rbrace$.
\end{demohigh}

\begin{demohigh}
La solution de $\systeme{x+2y=-5,4x+8y=1}$ est $\mathcal{S}=%
\left\lbrace \SolutionSystemeLUA(1,2;4,8)(-5,1) \right\rbrace$.
\end{demohigh}

\begin{demohigh}
La solution de $\systeme{-9x-8y=-8,3x-6y=-7}$ est $\mathcal{S}=%
\left\lbrace \SolutionSystemeLUA*[d](-9,-8;3,-6)(-8,-7) \right\rbrace$.
\end{demohigh}

\begin{demohigh}
La solution de $\systeme{x+y+z=-1,3x+2y-z=6,-x-y+2z=-5}$ est $\mathcal{S}=%
\left\lbrace \SolutionSystemeLUA(1,1,1;3,2,-1;-1,-1,2)(-1,6,-5) \right\rbrace$.
\end{demohigh}

\begin{demohigh}
La solution de $\systeme{x+y+z=-1,3x+2y-z=-5,-x-y+2z=0}$ est donnée par $X=%
\SolutionSystemeLUA*[d]<cell-space-limits=2pt>(1,1,1;3,2,-1;-1,-1,2)(-1,-5,0)[Matrice]$.
\end{demohigh}

\begin{demohigh}
La solution de $\systeme[xyzt]{y+z+t=1,x+z+t=-1,x+y+t=1,x+y+z=0}$ est $\mathcal{S}=%
\left\lbrace\SolutionSystemeLUA*[d](0,1,1,1;1,0,1,1;1,1,0,1;1,1,1,0)(1,-1,1,0)\right\rbrace$.
\end{demohigh}

\begin{demohigh}
La solution de $\systeme[xyzt]{x+2y+3z+4t=-10,5x+6y+7z=0,x+y+z+t=4,-2x-3y-5z-6t=7}$ est $\mathcal{S}=
\left\lbrace
\SolutionSystemeLUA
    [dec]<cell-space-limits=2pt>
    (1,2,3,4;5,6,7,0;1,1,1,1;-2,-3,-5,-6)(-10,0,4,7)
    [Matrice]
\right\rbrace$.
\end{demohigh}

\end{document}