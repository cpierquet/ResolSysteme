% !TeX TXS-program:compile = txs:///arara
% arara: lualatex: {shell: yes, synctex: no, interaction: batchmode}
% arara: lualatex: {shell: yes, synctex: no, interaction: batchmode} if found('log', '(undefined references|Please rerun|Rerun to get)')

\documentclass[a4paper,11pt]{article}
\def\TPversion{0.1.1}
\def\TPdate{7 Février 2023}
\usepackage[executable=python.exe]{pyluatex}
\usepackage[table,svgnames]{xcolor}
\usepackage{amsmath,amssymb}
\usepackage[bold-style=ISO,math-style=french]{unicode-math}
\setmainfont{TeX Gyre Schola}
\setmathfont{TeX Gyre Schola Math}
\usepackage[pyluatex]{ResolSysteme}
\usepackage{awesomebox}
\usepackage{fontawesome5}
\usepackage{systeme}
\usepackage{enumitem}
\usepackage{tabularray}
\usepackage{multicol}
\usepackage{fancyvrb}
\usepackage{fancyhdr}
\sisetup{locale=FR,output-decimal-marker={,}}
\fancyhf{}
\renewcommand{\headrulewidth}{0pt}
\lfoot{\sffamily\small [ResolSysteme]}
\cfoot{\sffamily\small - \thepage{} -}
\rfoot{\hyperlink{matoc}{\small\faArrowAltCircleUp[regular]}}

%\usepackage{hvlogos}
\usepackage{hologo}
\providecommand\tikzlogo{Ti\textit{k}Z}
\providecommand\TeXLive{\TeX{}Live\xspace}
\providecommand\PSTricks{\textsf{PSTricks}\xspace}
\let\pstricks\PSTricks
\let\TikZ\tikzlogo
\newcommand\TableauDocumentation{%
	\begin{tblr}{width=\linewidth,colspec={X[c]X[c]X[c]X[c]X[c]X[c]},cells={font=\sffamily}}
		{\huge \LaTeX} & & & & &\\
		& {\huge \hologo{pdfLaTeX}} & & & & \\
		& & {\huge \hologo{LuaLaTeX}} & & & \\
		& & & {\huge \TikZ} & & \\
		& & & & {\huge \TeXLive} & \\
		& & & & & {\huge \hologo{MiKTeX}} \\
	\end{tblr}
}

\usepackage{hyperref}
\urlstyle{same}
\hypersetup{pdfborder=0 0 0}
\usepackage[margin=1.5cm]{geometry}
\setlength{\parindent}{0pt}
\definecolor{LightGray}{gray}{0.9}

\usepackage[french]{babel}

\usepackage[most]{tcolorbox}
\tcbuselibrary{minted}
\NewTCBListing{PresentationCode}{ O{blue} m }{%
	sharp corners=downhill,enhanced,arc=12pt,skin=bicolor,%
	colback=#1!1!white,colframe=#1!75!black,colbacklower=white,%
	attach boxed title to top right={yshift=-\tcboxedtitleheight},title=Code \LaTeX,%
	boxed title style={%
		colframe=#1!75!black,colback=#1!15!white,%
		,sharp corners=downhill,arc=12pt,%
	},%
	top=\baselineskip,%
	fonttitle=\color{#1!90!black}\itshape\ttfamily\footnotesize,%
	listing engine=minted,minted style=colorful,
	minted language=tex,minted options={tabsize=4,fontsize=\small,autogobble},
	#2
}

\NewTCBListing{PresentationCodePython}{ O{DarkRed} m }{%
	sharp corners=downhill,enhanced,arc=12pt,skin=bicolor,%
	colback=#1!1!white,colframe=#1!75!black,colbacklower=white,%
	attach boxed title to top right={yshift=-\tcboxedtitleheight},title=Code Python,%
	boxed title style={%
		colframe=#1!75!black,colback=#1!15!white,%
		,sharp corners=downhill,arc=12pt,%
	},%
	fonttitle=\color{#1!90!black}\itshape\ttfamily\footnotesize,%
	listing engine=minted,minted style=colorful,
	minted language=python,minted options={tabsize=2,fontsize=\footnotesize,autogobble},
	#2
}

\newcommand\Cle[1]{{\bfseries\sffamily\textlangle #1\textrangle}}

\begin{document}

\setlength{\aweboxleftmargin}{0.07\linewidth}
\setlength{\aweboxcontentwidth}{0.93\linewidth}
\setlength{\aweboxvskip}{8pt}

\pagestyle{fancy}

\thispagestyle{empty}

\vspace{2cm}

\begin{center}
	\begin{minipage}{0.75\linewidth}
	\begin{tcolorbox}[colframe=yellow,colback=yellow!15]
		\begin{center}
			\begin{tabular}{c}
				{\Huge \texttt{ResolSysteme [fr]}}\\
				\\
				{\LARGE Des outils pour des} \\
				\\
				{\LARGE systèmes linéaires,} \\
				\\
				{\LARGE avec xint ou pyluatex.} \\
			\end{tabular}
			
			\bigskip
			
			{\small \texttt{Version \TPversion{} -- \TPdate}}
		\end{center}
	\end{tcolorbox}
\end{minipage}
\end{center}

\vspace{0.5cm}

\begin{center}
	\begin{tabular}{c}
	\texttt{Cédric Pierquet}\\
	{\ttfamily c pierquet -- at -- outlook . fr}\\
	\texttt{\url{https://github.com/cpierquet/ResolSysteme}}
\end{tabular}
\end{center}

\vspace{0.25cm}

{$\blacktriangleright$~~Des commandes pour travailler sur des matrices carrées (2x2, 3x3 ou 4x4).}

\smallskip

{$\blacktriangleright$~~Des commandes pour résoudre des systèmes linéaires (2x2, 3x3 ou 4x4).}

\smallskip

\vspace{1cm}

\begin{center}
	\begin{tcolorbox}[enhanced,colframe=ForestGreen,colback=lightgray!5,center,width=0.95\linewidth,drop fuzzy shadow=lightgray]
	Le \textbf{déterminant} de $A=\begin{pNiceMatrix} -1&{0,5} \\ \frac12&4 \end{pNiceMatrix}$ est
	$\det(A)=\DetMatrice[dec](-1,0.5 § 1/2,4)$.
	
	\medskip
	
	L'\textbf{inverse} de la matrice $A=\begin{pNiceMatrix} 1&2&3&4\\5&6&7&0\\1&1&1&1\\-2&-3&-5&-6 \end{pNiceMatrix}$ est $A^{-1}=\MatriceInversePY*[n]<cell-space-limits=2pt>(1,2,3,4 § 5,6,7,0 § 1,1,1,1 § -2,-3,-5,-6)$.
	
	\medskip
	
	La \textbf{solution} de $\systeme[xyzt]{y+z+t=1,x+z+t=-1,x+y+t=1,x+y+z=0}$ est $\mathcal{S}=\left\lbrace\SolutionSystemePY*[d](0,1,1,1 § 1,0,1,1 § 1,1,0,1 § 1,1,1,0)(1,-1,1,0)\right\rbrace$.
	\end{tcolorbox}
\end{center}

\vspace{0.5cm}

%\hfill{}\textit{Merci aux membres du groupe \faFacebook{} du \og Coin \LaTeX{} \fg{} pour leur aide et leurs idées !}

%\hfill{}\textit{Merci à Denis Bitouzé et à Patrick Bideault pour leurs retours et idées !}

\vfill

\hrule

\medskip

\TableauDocumentation

\medskip

\hrule

\medskip

\newpage

\phantomsection
\hypertarget{matoc}{}

\tableofcontents

\newpage

\part{Introduction}

\section{Le package ResolSysteme}

\subsection{Introduction}

\begin{noteblock}
L'idée est de \textit{proposer} des outils pour travailler sur des systèmes linéaires (de taille réduite !) :

\begin{itemize}
	\item en affichant la \textbf{solution} (si elle existe) ;
	\item en affichant le \textbf{déterminant} et l'éventuelle \textbf{inverse} de la matrice des coefficients.
\end{itemize}
\vspace*{-\baselineskip}\leavevmode
\end{noteblock}

\begin{importantblock}
À noter que les calculs -- en interne -- peuvent être effectués de deux manières :

\begin{itemize}
	\item via les packages \textsf{xint*} pour des formats \textbf{2x2} ou \textbf{3x3} ;
	\item via \textsf{python} et le package \textsf{pyluatex} (à charger manuellement du fait des options spécifiques) pour des formats \textbf{2x2}, \textbf{3x3} ou \textbf{4x4}.
\end{itemize}

Il n'est pas prévu -- pour le moment -- de travailler sur des matrices/systèmes plus grands.
\end{importantblock}

\begin{warningblock}
L'utilisation de \textsf{pyluatex} nécessite une compilation adaptée, à savoir en \hologo{LuaLaTeX} et en activant le mode \textsf{--shell-escape}. 

\smallskip

La méthode par \textsf{python} utilise quoi qu'il en soit le module \texttt{sympy}, qui doit donc être installé !
\end{warningblock}

\subsection{Packages utilisés, choix de formatage}

\begin{noteblock}
Le package charge les packages suivantes :

\begin{itemize}
	\item \texttt{xintexpr} et \texttt{xinttools} ;
	\item \texttt{sinuitx}, \texttt{nicefrac} et \texttt{nicematrix} ;
	\item \texttt{xstring} et \texttt{listofitems}.
\end{itemize}

Il est compatible avec les compilations usuelles en \textsf{latex}, \textsf{pdflatex}, \textsf{lualatex} (obligatoire pour \textsf{pyluatex} !!) ou \textsf{xelatex}.
\end{noteblock}

\begin{importantblock}
Les nombres sont formatés par la commande \texttt{\textbackslash num} de \textsf{sinuitx}, donc les options choisies par l'utilisateur se propageront aux résultats numériques.

\smallskip

L'affichage des matrices est gérée par le package \textsf{nicematrix}, et des options spécifiques \textit{simples} pourront être placées dans les différentes commandes.
\end{importantblock}

\subsection{Chargement du package, et option}

\begin{importantblock}
Le package peut donc se charger de deux manières différentes, suivant si l'utilisateur utilise \textsf{python} ou non. Les commandes \textit{classiques} sont disponibles même si \textsf{python} est utilisé.
\end{importantblock}

\begin{PresentationCode}{listing only}
%chargement du package sans passer par pyluatex, calculs via xint
\usepackage{ResolSysteme}
\end{PresentationCode}

\begin{PresentationCode}{listing only}
%chargement du package pyluatex et du package avec [pyluatex]
\usepackage[options]{pyluatex}
\usepackage[pyluatex]{ResolSysteme}
\end{PresentationCode}

\pagebreak

\part{Commandes}

\section{Une commande interne : écriture sous forme d'une fraction}

\subsection{La commande}

\begin{cautionblock}
En \textit{interne}, le code utilise une commande pour formater un résultat sous forme fractionnaire, avec gestion des entiers et gestion du signe \og $-$ \fg.
\end{cautionblock}

\begin{PresentationCode}{listing only}
\ConvVersFrac(*)[option de formatage]{calcul}
\end{PresentationCode}

\subsection{Utilisation}

\begin{tipblock}
Concernant cette commande, qui est dans un bloc \texttt{ensuremath} :

\begin{itemize}
	\item la version \textit{étoilée} force l'écriture du signe \og $-$ \fg{} avant l'éventuelle fraction ;
	\item le premier argument, \textit{optionnel} et entre \textsf{[...]} permet de spécifier un formatage du résultat :
	\begin{itemize}
		\item \Cle{t} pour l'affichage de la fraction en mode \textsf{tfrac} ;
		\item \Cle{d} pour l'affichage de la fraction en mode \textsf{dfrac} ;
		\item \Cle{n} pour l'affichage de la fraction en mode \textsf{nicefrac} ;
		\item \Cle{dec} pour l'affichage du résultat en mode \texttt{décimal} (sans arrondi !) ;
		\item \Cle{dec=k} pour l'affichage du résultat en mode \texttt{décimal} arrondi à $10^{-k}$ ;
	\end{itemize}
	\item le second argument, \textit{obligatoire}, est quant à lui, le calcul en syntaxe \textsf{xint}.
\end{itemize}
\vspace*{-\baselineskip}\leavevmode
\end{tipblock}

\begin{PresentationCode}{listing only}
\ConvVersFrac{-10+1/3*(-5/16)}          %sortie par défaut (fraction avec - sur numérateur)

\ConvVersFrac*{-10+1/3*(-5/16)}         %sortie avec - avant la fraction

\ConvVersFrac*[d]{-10+1/3*(-5/16)}      %sortie en displaystyle

\ConvVersFrac[n]{-10+1/3*(-5/16)}       %sortie en nicefrac

\ConvVersFrac[dec=4]{-10+1/3*(-5/16)}   %sortie en décimal arrondi à 0,0001

\ConvVersFrac{2+91/7}                   %entier correctement formaté
\end{PresentationCode}

\begin{PresentationCode}{text only}
\hfill\ConvVersFrac{-10+1/3*(-5/16)} \qquad
\ConvVersFrac*{-10+1/3*(-5/16)} \qquad
\ConvVersFrac*[d]{-10+1/3*(-5/16)} \qquad
\ConvVersFrac[n]{-10+1/3*(-5/16)} \qquad
\ConvVersFrac[dec=4]{-10+1/3*(-5/16)} \qquad
\ConvVersFrac{2+91/7}\hfill~
\end{PresentationCode}

\subsection{Interaction avec les commandes \og matricielles \fg, limitations}

\begin{warningblock}
En \textit{interne}, le formatage des résultats est géré par cette commande, et les options disponibles existent donc de la même manière pour les commandes liées aux systèmes linéaires.

\smallskip

Il ne sera par contre pas possible de spécifier des options différentes pour chacun des coefficients, autrement dit l'éventuelle option se propagera sur l'ensemble des résultats !

\smallskip

Les \textit{transformations} en fraction devraient pouvoir fonctionner avec des calculs \textit{classiques}, mais il est possible que, dans des cas \textit{spécifiques}, les résultats ne soient pas ceux attendus !
\end{warningblock}

\pagebreak

\section{Calcul de déterminant}

\subsection{Introduction}

\begin{cautionblock}
La première commande (matricielle) disponible est pour calculer le déterminant d'une matrice :

\begin{itemize}
	\item \textbf{2x2} ou \textbf{3x3} pour le package \textit{classique} ;
	\item \textbf{2x2} ou \textbf{3x3} ou \textbf{4x4} pour le package \textit{lua}.
\end{itemize}
\vspace*{-\baselineskip}\leavevmode
\end{cautionblock}

\begin{PresentationCode}{listing only}
%version classique
\DetMatrice(*)[option de formatage](matrice)

%version python
\DetMatricePY(*)[option de formatage](matrice)
\end{PresentationCode}

\subsection{Utilisation}

\begin{tipblock}
Concernant cette commande, qui est à insérer dans un environnement \textit{math} :

\begin{itemize}
	\item la version \textit{étoilée} force l'écriture du signe \og $-$ \fg{} avant l'éventuelle fraction ;
	\item le premier argument, \textit{optionnel} et entre \textsf{[...]} permet de spécifier un formatage du résultat :
	\begin{itemize}
		\item \Cle{t} pour l'affichage de la fraction en mode \textsf{tfrac} ;
		\item \Cle{d} pour l'affichage de la fraction en mode \textsf{dfrac} ;
		\item \Cle{n} pour l'affichage de la fraction en mode \textsf{nicefrac} ;
		\item \Cle{dec} pour l'affichage du résultat en mode \texttt{décimal} (sans arrondi !) ;
		\item \Cle{dec=k} pour l'affichage du résultat en mode \texttt{décimal} arrondi à $10^{-k}$ ;
	\end{itemize}
	\item le second argument, \textit{obligatoire} et entre \textsf{(...)}, est quant à lui, la matrice donnée par ses coefficients \textsf{a11,a12,... § a21,a22,...} (syntaxe \textit{héritée} de \texttt{sympy}).
\end{itemize}
\vspace*{-\baselineskip}\leavevmode
\end{tipblock}

\begin{PresentationCode}{}
%version classique
Le dét. de $A=\begin{pNiceMatrix}1&2\\3&4\end{pNiceMatrix}$ est
$\det(A)=\DetMatrice(1,2 § 3,4)$.
\end{PresentationCode}

\begin{PresentationCode}{}
%version classique
Le dét. de $A=\begin{pNiceMatrix}-1&{0,5}\\\frac12&4\end{pNiceMatrix}$ est
$\det(A)=\DetMatrice[dec](-1,0.5 § 1/2,4)$.
\end{PresentationCode}

\begin{PresentationCode}{}
%version classique
Le dét. de $A=\begin{pNiceMatrix}-1&\frac13&4\\\frac13&4&-1\\-1&0&0\end{pNiceMatrix}$ est
$\det(A) \approx \DetMatrice[dec=3](-1,1/3,4 § 1/3,4,-1 § -1,0,0)$.
\end{PresentationCode}

\begin{PresentationCode}{}
%version python
Le dé. de $A=\begin{pNiceMatrix}1&2\\3&4\end{pNiceMatrix}$ est
$\det(A)=\DetMatricePY(1,2 § 3,4)$.
\end{PresentationCode}

\begin{PresentationCode}{}
Le dét. de $A=\begin{pNiceMatrix}-1&{0,5}\\\frac12&4\end{pNiceMatrix}$ est
$\det(A)=\DetMatricePY*[d](-1,0.5 § 1/2,4)$.\end{PresentationCode}

\begin{PresentationCode}{}
%version python
Le dét. de $A=\begin{pNiceMatrix}-1&\frac13&4\\\frac13&4&-1\\-1&0&0\end{pNiceMatrix}$ est
$\det(A) \approx \DetMatricePY[dec=3](-1,1/3,4 § 1/3,4,-1 § -1,0,0)$.
\end{PresentationCode}

\section{Inverse d'une matrice}

\subsection{Introduction}

\begin{cautionblock}
La deuxième commande (matricielle) disponible est pour calculer l'éventuelle inverse d'une matrice :

\begin{itemize}
	\item \textbf{2x2} ou \textbf{3x3} pour le package \textit{classique} ;
	\item \textbf{2x2} ou \textbf{3x3} ou \textbf{4x4} pour le package \textit{lua}.
\end{itemize}
\vspace*{-\baselineskip}\leavevmode
\end{cautionblock}

\begin{PresentationCode}{listing only}
%version classique
\MatriceInverse(*)[option de formatage]<options nicematrix>(matrice)

%version python
\MatriceInversePY(*)[option de formatage]<options nicematrix>(matrice)
\end{PresentationCode}

\subsection{Utilisation}

\begin{tipblock}
Concernant cette commande, qui est à insérer dans un environnement \textit{math} :

\begin{itemize}
	\item la version \textit{étoilée} force l'écriture du signe \og $-$ \fg{} avant l'éventuelle fraction ;
	\item le premier argument, \textit{optionnel} et entre \textsf{[...]} permet de spécifier un formatage du résultat :
	\begin{itemize}
		\item \Cle{t} pour l'affichage de la fraction en mode \textsf{tfrac} ;
		\item \Cle{d} pour l'affichage de la fraction en mode \textsf{dfrac} ;
		\item \Cle{n} pour l'affichage de la fraction en mode \textsf{nicefrac} ;
		\item \Cle{dec} pour l'affichage du résultat en mode \texttt{décimal} (sans arrondi !) ;
		\item \Cle{dec=k} pour l'affichage du résultat en mode \texttt{décimal} arrondi à $10^{-k}$ ;
	\end{itemize}
	\item le deuxième argument, \textit{optionnel} et entre \textsf{<...>} correspond aux \Cle{options} à passer à l'environnement \texttt{pNiceMatrix} ;
	\item le troisième argument, \textit{obligatoire} et entre \textsf{(...)}, est quant à lui, la matrice donnée par ses coefficients \textsf{a11,a12,... § a21,a22,...} (syntaxe héritée de \texttt{sympy}).
\end{itemize}
À noter que si la matrice n'est pas inversible, le texte \texttt{Matrice non inversible} est affiché.
\end{tipblock}

\begin{PresentationCode}{}
%version classique
L'inverse de $A=\begin{pNiceMatrix}1&2\\3&4\end{pNiceMatrix}$ est
$A^{-1}=\MatriceInverse<cell-space-limits=2pt>(1,2 § 3,4)$.
\end{PresentationCode}

\begin{PresentationCode}{}
%version classique
L'inverse de $A=\begin{pNiceMatrix}1&2&3\\4&5&6\\7&8&8\end{pNiceMatrix}$ est
$A^{-1}=\MatriceInverse[n]<cell-space-limits=2pt>(1,2,3 § 4,5,6 § 7,8,8)$.
\end{PresentationCode}

\begin{PresentationCode}{}
%version python
L'inverse de $A=\begin{pNiceMatrix}1&2\\3&4\end{pNiceMatrix}$ est
$A^{-1}=\MatriceInversePY*[d]<cell-space-limits=2pt>(1,2 § 3,4)$.
\end{PresentationCode}

\begin{PresentationCode}{}
%version python
L'inv. de $A=\begin{pNiceMatrix}1&2&3&4\\5&6&7&0\\1&1&1&1\\-2&-3&-5&-6\end{pNiceMatrix}$ est
$A^{-1}=
\MatriceInversePY*[n]<cell-space-limits=2pt>(1,2,3,4 § 5,6,7,0 § 1,1,1,1 § -2,-3,-5,-6)$.
\end{PresentationCode}

\section{Résolution d'un système linéaire}

\subsection{Introduction}

\begin{cautionblock}
La deuxième commande (matricielle) disponible est pour déterminer l'éventuelle solution d'un système linéaire  qui s'écrit matriciellement $A\times X=B$:

\begin{itemize}
	\item \textbf{2x2} ou \textbf{3x3} pour le package \textit{classique} ;
	\item \textbf{2x2} ou \textbf{3x3} ou \textbf{4x4} pour le package \textit{lua}.
\end{itemize}
\vspace*{-\baselineskip}\leavevmode
\end{cautionblock}

\begin{PresentationCode}{listing only}
%version classique
\SolutionSysteme(*)[opt de formatage]<opts nicematrix>(matriceA)(matriceB)[opt Matrice]

%version python
\SolutionSystemePY(*)[opt de formatage]<opts nicematrix>(matriceA)(matriceB)[opt Matrice]
\end{PresentationCode}

\subsection{Utilisation}

\begin{tipblock}
Concernant cette commande, qui est à insérer dans un environnement \textit{math} :

\begin{itemize}
	\item la version \textit{étoilée} force l'écriture du signe \og $-$ \fg{} avant l'éventuelle fraction ;
	\item le premier argument, \textit{optionnel} et entre \textsf{[...]} permet de spécifier un formatage du résultat :
	\begin{itemize}
		\item \Cle{t} pour l'affichage de la fraction en mode \textsf{tfrac} ;
		\item \Cle{d} pour l'affichage de la fraction en mode \textsf{dfrac} ;
		\item \Cle{n} pour l'affichage de la fraction en mode \textsf{nicefrac} ;
		\item \Cle{dec} pour l'affichage du résultat en mode \texttt{décimal} (sans arrondi !) ;
		\item \Cle{dec=k} pour l'affichage du résultat en mode \texttt{décimal} arrondi à $10^{-k}$ ;
	\end{itemize}
	\item le deuxième argument, \textit{optionnel} et entre \textsf{<...>} correspond aux \Cle{options} à passer à l'environnement \texttt{pNiceMatrix} ;
	\item le troisième argument, \textit{obligatoire} et entre \textsf{(...)}, est quant à lui, la matrice $A$ donnée par ses coefficients \textsf{a11,a12,... § a21,a22,...} (syntaxe héritée de \texttt{sympy}) ;
	\item le quatrième argument, \textit{obligatoire} et entre \textsf{(...)}, est quant à lui, la matrice $B$ donnée par ses coefficients \textsf{b11,b21,...} (syntaxe héritée de \texttt{sympy}) ;
	\item le dernier argument, \textit{optionnel} et entre \textsf{[...]}, permet -- grâce à \Cle{Matrice} -- de présenter le vecteur solution.
\end{itemize}
À noter que si la matrice n'est pas inversible, le texte \texttt{Matrice non inversible} est affiché.
\end{tipblock}

\begin{PresentationCode}{}
%version classique
La solution de $\systeme{3x+y-2z=-1,2x-y+z=4,x-y-2z=5}$ est $\mathcal{S}=%
\left\lbrace \SolutionSysteme*[d](3,1,-2 § 2,-1,1 § 1,-1,-2)(-1,4,5) \right\rbrace$.\\
\end{PresentationCode}

\begin{PresentationCode}{}
%version python
La solution de $\systeme{x+y+z=-1,3x+2y-z=6,-x-y+2z=-5}$ est $\mathcal{S}=%
\left\lbrace \SolutionSystemePY(1,1,1 § 3,2,-1 § -1,-1,2)(-1,6,-5) \right\rbrace$.
\end{PresentationCode}

\begin{PresentationCode}{}
%version python
La solution de $\systeme[xyzt]{x+2y+3z+4t=-10,5x+6y+7z=0,x+y+z+t=4,-2x-3y-5z-6t=7}$
est $\mathcal{S}=%
\left\lbrace
	\SolutionSystemePY%
		[dec]<cell-space-limits=2pt>%
		(1,2,3,4 § 5,6,7,0 § 1,1,1,1 § -2,-3,-5,-6)(-10,0,4,7)%
		[Matrice]
\right\rbrace$.
\end{PresentationCode}

\begin{PresentationCode}{}
%pas de solution
La solution de $\systeme{x+2y=-5,4x+8y=1}$ est $\mathcal{S}=%
\left\lbrace \SolutionSystemePY(1,2 § 4,8)(-5,1) \right\rbrace$.
\end{PresentationCode}

\pagebreak

\part{Fonctions python utilisées}

\begin{cautionblock}
Les fonctions utilisées par les packages \textsf{pyluatex} ou \textsf{pythontex} sont données ci-dessous.

Elles sont accessibles en \textit{natif} une fois l'option \textsf{lua} activée, grâce notamment à la macro \texttt{\textbackslash py}.
\end{cautionblock}

\begin{PresentationCodePython}{listing only}
import sympy as sy
x = sy.Symbol('x')
y = sy.Symbol('y')
z = sy.Symbol('z')
t = sy.Symbol('t')

def resol_systeme_QQ(a,b,c,d,e,f,g,h,i,j,k,l,m,n,o,p,q,r,s,u) :
	solution=sy.solve([a*x+b*y+c*z+d*t-e,f*x+g*y+h*z+i*t-j,k*x+l*y+m*z+n*t-o,p*x+q*y+r*z+s*t-u],[x,y,z,t])
	return solution

def resol_systeme_TT(a,b,c,d,e,f,g,h,i,j,k,l) :
	solution=sy.solve([a*x+b*y+c*z-d,e*x+f*y+g*z-h,i*x+j*y+k*z-l],[x,y,z])
	return solution

def resol_systeme_DD(a,b,c,d,e,f) :
	solution=sy.solve([a*x+b*y-c,d*x+e*y-f],[x,y])
	return solution

def det_matrice_QQ(a,b,c,d,e,f,g,h,i,j,k,l,m,n,o,p) :
	MatTmp = sy.Matrix(([a,b,c,d],[e,f,g,h],[i,j,k,l],[m,n,o,p]))
	DetMatTmp = MatTmp.det()
	return DetMatTmp

def det_matrice_TT(a,b,c,d,e,f,g,h,i) :
	MatTmp = sy.Matrix(([a,b,c],[d,e,f],[g,h,i]))
	DetMatTmp = MatTmp.det()
	return DetMatTmp

def det_matrice_DD(a,b,c,d) :
	MatTmp = sy.Matrix(([a,b],[c,d]))
	DetMatTmp = MatTmp.det()
	return DetMatTmp

def inverse_matrice_QQ(a,b,c,d,e,f,g,h,i,j,k,l,m,n,o,p) :
	MatTmp = sy.Matrix(([a,b,c,d],[e,f,g,h],[i,j,k,l],[m,n,o,p]))
	DetMatTmp = MatTmp.inv()
	return DetMatTmp

def inverse_matrice_DD(a,b,c,d) :
	MatTmp = sy.Matrix(([a,b],[c,d]))
	InvMatTmp = MatTmp.inv()
	return InvMatTmp

def inverse_matrice_TT(a,b,c,d,e,f,g,h,i) :
	MatTmp = sy.Matrix(([a,b,c],[d,e,f],[g,h,i]))
	InvMatTmp = MatTmp.inv()
	return InvMatTmp
\end{PresentationCodePython}

\pagebreak

\part{Historique}

\verb|v0.1.1|~:~~~~Correction d'un bug avec le caractère \og ; \fg{}

\verb|v0.1.0|~:~~~~Version initiale

\end{document}